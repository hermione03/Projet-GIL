\documentclass[12pt,a4paper]{moderncv}
\moderncvtheme[purple]{classic}
\usepackage[utf8]{inputenc}
\usepackage[inline]{enumitem}
% Marge aux 4 coins de la page, ici elles sont réduites pour gagner de la place
\usepackage[top=1.0cm, bottom=1.0cm, left=1.6cm, right=1.6cm]{geometry}
% Largeur de la colonne de gauche pour les dates
\setlength{\hintscolumnwidth}{2.7cm}
\firstname{Maria}
\familyname{Messaoud-Nacer}
\title{Etudiante en informatique}
\address{2 rue qui n'existe pas}{93200 saint-denis}
\email{maria.messaoudnacer03@gmail.com}
\mobile{06 00 00 00 00}
\extrainfo{18 ans}
\begin{document}
\maketitle
% Marge négative entre le titre et la partie expérience, pour gagner de la place
\vspace*{-2.5\baselineskip}

\section{Formations}
\cventry{Septembre 2021 -- À aujourd'hui}{Premiere Année de licence }{Université de Paris 8}{}{}{Licence informatique}
\cventry{Décembre 2020 -- Juin 2021}{Premiere Année de licence }{Universite des sciences et technologies Houari Boumedien(USTHB)}{Bab-Ezzouar , Alger}{Algerie}{Tronc-commun Mathematiques Informatique}
\section{Certifications}
\cventry{Octobre 2020}{Diplome du Baccalauréat}{Série Mathematiques}{{\underline{Mention}}:Bien}{{\underline{Moyenne}}: 15,71}.
\cventry{Janvier 2020}{TCF DAP}{{\underline{Niveau Global}}: C2}{Valable 2 ans}{}.

\section{Compétences en informatique}
\cvdoubleitem{\underline{OS}}{Debian, Ubuntu, PopOS}{}{}
\cvitem{\underline{Programmation}}{HTML, CSS, Python, C, Latex, Godot Visual scripting }
\cvitem{\underline{Projets:}}{Jeu Godot Engine, Text to HTML(Python), Page web(HTML CSS )}
\section{Compétences Linguistiques}
\cvlanguage{\underline{Anglais}}{lu, écrit, parlé}{Niveau: Moyen}
\cvlanguage{\underline{Français}}{lu, écrit, parlé}{Niveau: Trés Bien}
\cvlanguage{\underline{Arabe}}{lu, écrit, parlé}{Niveau: Trés Bien}
\section{Centres d'intérêt}
\cvitem{}{Musique(Instruments divers), Sport, Casse-tetes en tout genre, Lecture, Voyages\dots}
\end{document}